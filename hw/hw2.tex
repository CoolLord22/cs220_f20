\documentclass[12pt]{article}
%% meta information
% JHU Fall 2020 CS 220 Intermediate Programming Homework 2
% Authors: Ali Darvish, Sing Chun Lee
% Contributors: David Hovemeyer, Joanne Selinski, Misha Kazhdan, Sara More, ... 

%% packages
\usepackage[margin=1in]{geometry}	% change page margins
\usepackage[dvipsnames]{xcolor}		% additional color names
% color names reference: https://en.wikibooks.org/wiki/LaTeX/Colors
\usepackage{fancyhdr} 				% customize header
\usepackage[tikz]{bclogo}			% additional symbols
% symbol reference: http://tug.ctan.org/info/symbols/comprehensive/symbols-letter.pdf
\usepackage{mdframed}				% use to create caution box
\usepackage{minted}					% code block colors
% **Note**: to use minted, need to add compile flag -shell-escape
\usemintedstyle{emacs}				% mint color style
% more styles on: https://www.overleaf.com/learn/latex/Code_Highlighting_with_minted
\usepackage{xparse}					% show inline code
\usepackage{realboxes}				% inline color box
\usepackage{mdwlist}				% suspend resume enumerate list

%% variables
% Homework Assignment Number
\def \HWNumber {2}
% Homework Assignment Due Date
\def \HWDueDate {Friday September 25\textsuperscript{th} at 11pm EDT (Baltimore time)}

%% new environment
\newenvironment{caution}{\par\begin{mdframed}[linewidth=1pt,linecolor=YellowOrange]%
		\begin{list}{}{\leftmargin=0cm}\item[\Large\bcdanger]}
		{\end{list}\end{mdframed}\par}
\newenvironment{danger}{\par\begin{mdframed}[linewidth=1pt,linecolor=Red]%
		\begin{list}{}{\leftmargin=0cm}\item[\Large\bcbombe]}
		{\end{list}\end{mdframed}\par}
\newenvironment{tip}{\par\begin{mdframed}[linewidth=1pt,linecolor=ForestGreen]%
		\begin{list}{}{\leftmargin=0cm}\item[\Large\bclampe]}
		{\end{list}\end{mdframed}\par}
\newenvironment{info}{\par\begin{mdframed}[linewidth=1pt,linecolor=Cerulean]%
		\begin{list}{}{\leftmargin=0cm}\item[\Large\bcinfo]}
		{\end{list}\end{mdframed}\par}
\newenvironment{answerbox}{\par\begin{mdframed}[linewidth=1pt,linecolor=Black]%
		\begin{list}{}{\leftmargin=0.5cm}\item[\Large\bcplume]}
		{\end{list}\end{mdframed}\par}
\newminted[c99]{c}{obeytabs=true,tabsize=4}	% new c style, changed tabsize to 4
\newenvironment{codeblock}{\VerbatimEnvironment\begin{c99*}{}}{\end{c99*}} % use the new c style as code block
	
%% new commands
\newcommand{\answer}[2]{
	\suspend{enumerate}
	\begin{answerbox}
		\textbf{ANSWER:} 
		
		\ifthenelse{\equal{#2}{}}{\vspace{#1}}{#2}
	\end{answerbox}
	\resume{enumerate}
}

\newcommand{\answerlast}[2]{
	\suspend{enumerate}
	\begin{answerbox} 
		\textbf{ANSWER:} 
		
		\ifthenelse{\equal{#2}{}}{\vspace{#1}}{#2}
	\end{answerbox}
}
	
%% headers
\pagestyle{fancy}
\fancyhf{}
\rhead{Fall 2020}
\chead{Intermediate Programming}
\lhead{EN.601.220}
\lfoot{\copyright2020 Johns Hopkins University}
\rfoot{Page \thepage}

%% student information variables
\def \SName {}
\def \JHEID {}

\begin{document}
% Student information boxes and the score box
\noindent Name: \framebox[5.5cm]{\SName\rule{0pt}{12pt}} \hfill
JHEID: \framebox[3cm]{\JHEID\rule{0pt}{12pt}}\hfill
Score: \framebox[2cm]{\rule{0pt}{12pt}}

\section*{Homework \HWNumber}

\begin{caution}
	\textbf{CAUTION}
	
	\begin{itemize}
		\item You are expected to work individually.
		\item \textbf{Due: \HWDueDate.}
		\item \textit{This assignment is worth 20 points. }
	\end{itemize}
\end{caution}

\begin{danger}
	\textbf{SUBMISSION REQUIREMENT}
	
	Answer each problem in this \textbf{pdf}, in the area to the side of the problem, or immediately after it.  You may either type your solutions, or hand-write and scan them in, but they need to be legible, and part of this document. If you need to add additional sheets, please make a note near the problem itself that the grader should "see attached". Submit the \textbf{pdf} document via GradeScope once you have added your answers.
\end{danger}

\subsection*{Learning Objectives}
\begin{tip}
	\textbf{OBJECTIVES}
	
	\begin{itemize}
		\item control flow
		\item c-style strings
		\item data types
		\item arrays
	\end{itemize}
\end{tip}

\begin{info}
	\textbf{INFO}
	
	Many problems make use of “code fragments", which can be thought of as pieces of code extracted from complete programs. While a code fragment will not generally compile by itself, we will assume that it exists in a sensible framework (i.e. is inside a properly formed \mintinline{c}|main()|, all appropriate headers and libraries have been included, etc.). We will also assume that there is no other code in the program that would impact the behavior of the fragment; each fragment is designed to be understood in isolation.
\end{info}

\newpage
\subsubsection*{Part I: Code Puzzles. [1 point each problem]}

Trace through each code fragment and write down the exact output that will be printed if the fragment is run, assuming it is embedded in a proper program with the necessary \mintinline{c}|#include| statements. If there is no output generated, write ``\textbf{no output}", and give one sentence explaining why.\\ 

\begin{tip}
	\textbf{TIP}
	
	Note that these are called ``puzzles" because their behavior may not be intuitive or correct (though the code itself is valid and will compile, albeit with warnings in some cases).  \textbf{\textit{If you think you have spotted a typo in Part I, it is intentional!}}
\end{tip}

\begin{enumerate}
% Q1
\item \begin{codeblock}
int i = 1;
while (i < 10); {
	if (i % 2 >= 0)
	printf("%d ", i++);
}
\end{codeblock}
\answer{2.8cm}{%
% Type your answers here	
}
% Q2
\item \begin{codeblock}
for (int i = 0 ; i < 5 ; i++) {
	for (int j = 0 ; j < 2 ; j++) {
		if( i == j + 1 ) { break; }
		else { printf( "%d %d\n" , i , j ); }
	}
}
\end{codeblock}
\answer{2.8cm}{%
% Type your answers here	
}
% Q3
\item \begin{codeblock}
int x = 0;
while (true) {
	printf("x = %d\n", ++x);
	if (x = 4) { break; }
}
\end{codeblock}
\answer{4cm}{%
% Type your answers here
}
% Q4
\item \begin{codeblock}
int i = 15;
while (i > 10) {
	int sum = i;
	sum = sum + i;
	printf ("%d\n", sum);
	i--;
}
\end{codeblock}
\answer{4cm}{%
% Type your answers here
}
% Q5
\item \begin{codeblock}
char str[] = "this is a test only!";
for(int i = 0; i < (int)strlen(str); i++) {
	printf("str[%d] = %c\n", i, str[i]);
	if(str[i] == ' ') { str[i] = 0; }
}
printf("strlen( %s ) = %d\n", str, (int)strlen(str));
\end{codeblock}
\answer{3cm}{%
% Type your answers here
}
% Q6
\item \begin{codeblock}
char c = 'A';
while (c >= 'a')
	printf("%c ", c--);
\end{codeblock}
\answer{3cm}{%
% Type your answers here
}
% Q7
\item \begin{codeblock}
int a = 1;
switch(a) {
	case '1':
		printf("ONE\n");
	break;
	case '2':
		printf("TWO\n");
	break;
	defaultt:
		printf("THREE\n");
}
\end{codeblock}
\answer{3cm}{%
% Type your answers here
}
% Q8
\item \begin{codeblock}
float f = 0.0f;
int i;
for(i = 0; i < 20; i++)
	f = f + 0.1f;
if (f == 2.0f)
	printf("f is 2.0 \n");
else
	printf("f is NOT 2.0\n");
\end{codeblock}
\answer{4cm}{%
% Type your answers here
}
% Q9
\item \begin{codeblock}
int i = 3;
if ((--i < 3) || (i--/4) || !(i-- > 2))
	printf("Hello\n");
if (i--)
	printf("Goodbye\n");
printf("%d\n", i);
\end{codeblock}
\answerlast{4cm}{%
% Type your answers here
}

\newpage

\subsubsection*{Part II: Code Correctness. [1 point each problem]}

Trace through the code fragments and explain what is wrong with them. You are not expected to show the output.

\resume{enumerate}
% Q10
\item \begin{codeblock}
// collect a valid face number of a deck
int face = 0;
do {
	char msg[] = "Please enter the face number [1-13]: "; 
	printf( "%s", msg);
	scanf("%d", &face);
} while (face >= 1 && face <= 13);
\end{codeblock}
\answer{5cm}{%
% Type your answers here
}
% Q11
\item \begin{codeblock}
float n;
printf("Enter a number: ");
scanf("%f", n);
printf("You entered %f \n", n);
\end{codeblock}
\answer{5cm}{%
% Type your answers here
}
% Q12
\item \begin{codeblock}
char source[] = "hello folks";
char destination[11];
strcpy(destination, source);
for(int i = 0; source[i]; i++) {
	printf("%c" , source[i]); 
}
printf("\n");
for(int i = 0; destination[i]; i++) {
	printf("%c" , destination[i]); 
}
printf("\n");
\end{codeblock}
\answerlast{6cm}{%
% Type your answers here
}

\subsubsection*{Part III: Code Reading. [2 points each problem]}

At a high level, explain what the following functions or code fragments do. The explanation should \textbf{\underline{not}} be a direct translation of the code statements.

\resume{enumerate}
% Q13
\item \begin{codeblock}
unsigned fun1(unsigned a, unsigned b) {
	int count = 0;
	int sum = b;
	while (sum <= a) {
		sum += b;
		++count;
	}
	return count;
}
\end{codeblock}
\answer{5.5cm}{%
% Type your answers here
}
% Q14
\item \begin{codeblock}
// Assume that "str" is assigned some string value
void fun2(char str[]) {
	for(unsigned int i = 0; i < strlen(str); i++) {
		if( str[i] >= 'A' && str[i] <= 'Z' ) { 
			str[i] -= 'A' - 'a'; 
		}
	}
	printf("%s\n", str);
}
\end{codeblock}
\answer{5.5cm}{%
% Type your answers here
}
% Q15
\item \begin{codeblock}
int arr[] = {0, 1, 2, 3, 4, 5, 6, 7, 8, 9};
int p = sizeof(arr) / sizeof(int) - 1;
int k = arr[p];
for (int j = p; j >= 1; j--)
	arr[j] = arr[j - 1];
arr[0] = k;
\end{codeblock}
\answer{5.5cm}{%
% Type your answers here
}
% Q16
\item \begin{codeblock}
// Assume "abs" is a defined/accessible function and
// returns the absolute value of what is passed into it   
int fun3(int a, int b) {
	return ((a + b) + abs(b - a)) / 2;
}
\end{codeblock}
\answerlast{5.5cm}{%
% Type your answers here
} 
% Note: suspend has already ended the environment

\begin{center}
	\vspace{1cm}
	\large \textless The END of Homework \HWNumber\textgreater
\end{center}

\end{document}
